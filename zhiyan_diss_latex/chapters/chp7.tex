 \chapter{Conclusion}
 \label{ch:7}

While much research has investigated the accentedness of non-native (L2) English speech, very few has attempted to rank individual phonetic and phonological patterns in L2 speech according to their perceived foreign accentedness. The current study contributes to the field of foreign accent by providing accentedness rankings of various phonetic patterns in L2 speech. And even while the previous studies in foreign accent that did focuss on specific phonetic patterns in L2 speech only dealt with a limited few, the current study investigated the perceived accentedness of a larger variety of phonetic patterns.

Based on empirical observations in Experiments 1 and 2, the current study further hypothesized that raters' knowledge of English phonetics and phonological has affected their acccentedness judgment. A Naïve Discriminative Learning (NDL) model was built to evaluate this hypothesis (Chapter \ref{ch:6}). The NDL model achieved moderate success in explaining rating differences between different types of L2 stimuli. This chapter will first summarize major findings of the current study and discuss their implications. The focus will then shift to limitations of the current study and recommendations for future research.

\section{Summary of Results}

The current study selected 100 L2 English speech samples from the Speech Accent Archive (SAA) for two perception experiments to elicit accentedness judgment on different types of phonetic patterns. A third experiment was conducted to investigate how raters’ L1 knowledge affects their accentedness judgement. In Experiments 1 and 2, native (L1) American English listeners (i.e., the raters) listened to and rated the foreign-accentedness of 100 L2 speech samples. Experiment 1 and 2 focused on the observation of accentedness ratings of the L2 speech samples in five phonological contexts, namely, “\textit{please call},” “\textit{ask her},” “\textit{five thick},” “\textit{six spoons}” and “\textit{small plastic}.” Phonetic transcriptions (IPA transcriptions) from 100 L1 American English speakers in the SAA were surveyed to find the most common L1 productions for the five contexts (e.g., [faɪv θɪk] for “\textit{ask her}”). 

L2 speech samples whose IPA transcriptions match the most common L1 productions were termed match stimuli. L2 speech samples whose IPA transcriptions that differ from the most common L1 productions were termed the mismatch stimuli. Among the mismatch stimuli, some differ from the most common L1 productions by only one consonant (e.g., [faɪv tɪk]), some differ from the most common L1 productions by only one vowel (e.g., [faɪv θik]), some differ from the most common L1 productions by only syllable structural element (e.g., [faɪvə θɪk]). These three types of stimuli were termed consonant mismatch, vowel mismatch, and syllable mismatch, respectively. Some of the mismatch stimuli contain phonetic patterns that were not observed in L1 speech (e.g., [faɪvə θɪk], [faɪv θik]); others contain phonetic patterns that exist in L1 speech (e.g., [faɪv tɪk]). Acoustic analysis was conducted to verify the reliability of the IPA transcriptions for the L2 speech samples. The current study accepted the IPA transcriptions as reliable because acoustic differences between the L2 speech samples and their corresponding L1 speech samples were captured by the IPA transcriptions for the L2 speech samples.


The results of the two perception experiments show that stimuli with consonant and syllable mismatches were judged as being more accented than stimuli with vowel mismatches. The three types of mismatch stimuli were judged as being more accented than the match stimuli. The types of mismatches that exist in L1 speech were judged as relatively less accented (e.g., [θ]$\rightarrow$[t]). The types of mismatches that do not exist in L1 speech were judged as relatively more accented (e.g., [θ]$\rightarrow$[s̪t̪]). These results show that raters were aware of which phonetic patterns are allowed in L1 speech. Analysis also show that phonological contexts could have affected the accentedness judgment on some types of mismatches. For example, VOT-shortening was rated as being more accented phrase-initially than phrase-medially. The two perception experiments therefore show that (1) different types of phonetic patterns in L2 speech do not carry equal weight in accentedness perception; (2) raters’ knowledge of L1 phonetics and phonology potentially affects accentedness judgment; and (3) phonological context potentially affects accentedness perception.


The two perception experiments differed in their respective experimental designs. Experiment 1 did not provide any training to the raters. In addition, the raters in Experiment 1 were not aware of the intended meanings of each stimulus. Unlike Experiment 1, Experiment 2 included a training phase, and it controlled for intelligibility by informing raters of the intended meaning of each stimulus. As a result, ratings obtained in Experiment 2 were more consistent than ratings obtained in Experiment 1. The stimuli were generally judged as more accented in Experiment 2 than in Experiment 1. Accentedness ratings of both Experiment 1 and Experiment 2 experienced a gradual increase during the entirety of the experiment. That is, the same stimulus might receive a higher rating (i.e., be perceived as being more accented) if it occurred later in the experiment. These results show that (1) a training phase might be valuable to familiarize raters with the procedure of the experiment and the range of the accents covered by the stimuli; (2) stimuli tend to be judged as being more accented once their intended meanings were known.

Results of the two perception studies indicate that raters’ L1 phonetic and phonological knowledge with regard to the five phonological contexts (L1 knowledge) could have affected their accentedness judgment. Experiment 3 attempted to directly investigate how raters’ judgments of accentedness were affected by their L1 knowledge. A Naïve Discriminative Learning Model (NDL) was employed to investigate raters’ L1 knowledge by examining the co-occurrences of pronunciations (e.g., [æsk]) and lexical outcomes (e.g., “\textit{ask}”) in the L1 speech of American English. The model assigned association strengths to trigram sequences (e.g., ``\textit{\#æs}," ``\textit{æsk}," ``\textit{sk\#}"), measuring the probability for each trigram to predict the intended lexical outcome (e.g., how probable it is for ``\textit{\#æs}" to predict “\textit{ask}”). Raters’ L1 knowledge was, therefore, approximated by using a matrix of association strengths, mapping trigram phonetic segment sequences to lexical outcomes.

The 100 L2 stimuli were subsequently evaluated against raters’ L1 knowledge to estimate the degree of dissimilarity (NDL-distance) between L1 and L2 speech samples. The results showed that NDL-distance correlates significantly with accentedness ratings obtained by Experiment 2, suggesting that L1 knowledge, as approximated by the NDL model, could have potentially affected accentedness judgments. Comparisons between linear mixed-effects regression models further revealed that rating differences between stimuli with consonant, vowel and syllable mismatches were not significant when NDL-distances were controlled for. These results show that rating differences between the different types of stimuli could be explained by the  association strengths from the stimuli to their intended lexical outcomes.

The NDL analysis reveals that consonants in the 100 L1 American English speakers’ speech do not vary as much as vowels. For example, L1 productions of the vowel in such a word as “\textit{small}” could be [ɑ, ɔ, a, ɑ̘ , ɑ̝, ɑ:, ɔ:, aʊ, ɑʊ̆], according to IPA transcriptions in the SAA. By comparision, the consonants in such a word as “\textit{small}” exhibit a considerably lower degree of variability. The NDL model, therefore, assigned higher association strengths to trigram cues involving consonants than to trigram cues involving vowels. Changing the consonants in a pronunciation is therefore more likely to lower its association strength to its lexical outcome. In other words, consonant changes are more likely to impair lexical identification. Previous studies in lexical identification claim that consonants are more important than vowels in identifying lexical outcomes \citep{Nespor_2003}. Findings of Experiment 3 support such a claim and further demonstrates that lexical identification potentially plays a role in accentedness judgment.

\section{Theoretical Implications and Societal Impacts}

The current study has important theoretical and applied implications. L1 English listeners’ perception of foreign accent reveals the nature of “foreign accentedness” and how the deviation from L1 grammar affects accentedness perception. Sociolinguistics research often reports that pronunciation, rather than vocabulary or syntax, is a major factor that affects communication \citep{Grant_2014}. Although L2 accents bear no relationship to one’s intelligence or personal character, they do carry a potential stigma that could cause negative social and workplace consequences \citep{Gluszek_2010}. Second language or foreign language learners, therefore, place great importance on the correctness of their pronunciation \citep{Waniek-Klimczak_2015}. However, many English language instructors are reluctant to incorporate pronunciation instruction into their teaching curriculum \citep{Thomson_2014}. One reason for such reluctance is that L2 pronunciation errors are numerous, and there is not enough time for teachers to address all of them \citep{Munro_2006, Thomson_2014}. 

By identifying phonetic patterns that are most accented to L1 English listeners, the current study could help language teachers set priorities for pronunciation instructions. The current investigation therefore could enable a much more efficient and perhaps effective strategy for pronunciation instructions and could more broadly help disadvantaged linguistic minorities to achieve their full potentials in society. The current research also endeavored to model L1 productions computationally and subsequently compare L2 productions against L1 productions. The model adopted by the current research potentially reveals the nature of “foreign accentedness” and could further help design improved speech analysis algorithms. 

\section{Discussion and Future Directions}

The current study based its analysis on IPA transcriptions rather than acoustic information of the selected speech samples. Therefore, the reliability of the IPA transcriptions was of utmost importance. Although transcribers of the SAA, from which data of the current study were extracted, were diligent in transcribing every speech sample, inaccuracies are still possible. The current study, therefore, examined the IPA transcriptions for the 100 selected stimuli by measuring benchmark acoustic signals. Results of the acoustic analysis have indeed partially validated the transcriptions, but arguments could still be made against the validity of the acoustic analysis itself. As stated in Chapter \ref{ch:3}, acoustic correlates of phonemes are multidimensional. The current study examined only one or two acoustic measurements of relevant segments, and this could have oversimplified the issue. 

On the other hand, acoustic analysis mostly concerns speech production, while IPA transcriptions in the SAA could be considered as a reflection of the transcribers' perception of the speech samples. To increase the reliability of the IPA transcriptions, the most direct approach is perhaps to recruit more trained transcribers. When there are more people transcribing the same speech samples, the most reliable transcriptions could then be determined via inter-transcriber agreement. We have undertaken such an endeavor. Preliminary results showed that 73\% of the transcriptions submitted by the 67 newly recruited transcribers matched the existing transcriptions in the Speech Accent Archive \citep{Weinberger_2019}. More trained transcribers are still needed to further improve the reliability of the IPA transcriptions.

Previous research in foreign accent often suggests that the degree of (dis)similarity between an L2 speech and its L1 counterparts is responsible for the degree of perceived foreign accentedness. The practical problem in comparing L1 and L2 speech samples is that L1 speech exhibits considerable within- and between-speaker variability. The term “native speaker norm” is often mentioned to refer to speech patterns of a hypothetical “average” L1 speaker. The current study adopted a similar concept in its selection of stimuli by classifying stimuli based on whether they match the most common L1 pronunciations in the SAA.

Pronunciations that differ from the most common L1 productions were considered as containing mismatches. The potential drawback is that some of the mismatches could be dialectal variations of L1 English (e.g., pronouncing “\textit{thick}” as [tɪk] or [fɪk]). These variations indeed received relatively lower accentedness ratings (i.e., less foreign accented). The general finding in Experiments 1 and 2 is that consonant mismatches are more accented than vowel mismatches. However, if more L1 dialectal vowel variations were included in the current study than L1 dialectal consonant variations, then the general finding should be that non-dialectal variations are more accented than dialectal variations. To investigate whether consonants are indeed more important than vowels in accentedness judgment, we need to consider how likely for a specific mismatch to exist in L1 speech. 

Experiment 3 further investigated how the frequency of occurrence of a phonetic/phonological pattern in L1 speech could affect the accentedness judgments on this phonetic/phonological pattern. The model assigned higher association strength to phonetic patterns that could be considered dialectal (i.e., spoken by a subset of the 100 L1 speakers). For example, only two out of the 100 L1 speakers pronounced “\textit{thick}” as [tɪk]. The association strength for [tɪk], as calculated by the NDL model, is 73.27\% rather than 2\%, showing that [tɪk] has a 73.27\% probability of predicting the lexical outcome “\textit{thick}.” Therefore, it was considered relatively more native-like. When the frequency of occurrence of a phonetic/phonological pattern in L1 speech was controlled for, the rating differences between consonant mismatches and vowel mismatches diminished. This finding demonstrates that consonant mismatches are not inherently more accented than vowel mismatches. Rating differences between the different types of stimuli could be explained via “association strengths” from phonetic segment sequences to their respective lexical outcomes.

Although the NDL model achieved moderate success in accounting for the data in the current study, the model was built on data from only 100 L1 speakers of American English, which are probably not representative enough. In addition to the omission of other L1 varieties of English, the 100 L1 speakers of American English were from 37 states and the District of Columbia. States such as Arizona, Colorado and Delaware were not represented (See Section \ref{dem:native} in Appendix \ref{ap:A} for detailed demographic information). To better model raters’ L1 knowledge, data from more L1 speakers are undoubtedly needed. 

With more data from L1 English speakers, the model could be tailored to approximate accentedness judgments of specific groups of listeners. For example, L1 British English listeners' accentedness judgments could be approximated by including more production data from L1 English varieties in the British Isles. Some research shows that the experience with a certain type of accent could potentially affect accentedness judgment \citep{wester_2014}. Mandarin listeners, for example, tend to judge Mandarin-accented English less accented than do L1 English listeners \citep{wester_2014}. The NDL model could therefore potentially approximate Mandarin speakers' accentedness judgment by including production data from Mandarin speakers of English.

The current study did not discuss sociolinguistic elements such as one’s own dialect and familiarity with L2 speech, both of which could potentially affect accentedness judgments. As shown in van den Doel (2006), British English speakers and American English speakers do not always agree on which phonetic patterns in L2 speech are accented. The current study focuses only on ratings from L1 listeners of American English. However, the L1 American English listeners' accentedness judgments are hardly homogeneous (See Appendix \ref{ap:C} for mean ratings from raters of each state). 

L1 listeners of southern American English or people who are familiar with southern American English might be more tolerant to monophthongizations such as [aɪ] to [a], because such sound change is similar to off-glide deletions in many varieties of southern American English \citep{Labov_2005}. Raters from regions with a large Hispanic population (e.g., California, Arizona and Texas) are very likely to have been exposed to Spanish-accented English, and thus could be more familiar with Spanish speakers’ L2 English speech patterns such as vowel prothesis of s-clusters. Due to the limited access to raters’ personal information, the current study could not warrant a detailed investigation on sociolinguistic factors. Future research is needed to further investigate how one’s sociolinguistic background affects his/her accentedness judgment. 

Overall, findings in this dissertation contribute to the field of foreign accent by providing accentedness rankings of different types of phonetic/phonological patterns in L2 speech. This disseration further investigated why these phonetic/phonological patterns were weighted differently in L1 accentedness judgment. Results show that L1 knowledge, as measured by statistical properties of phonetic segments in L1 English speech, could have contributed to the degree of perceived accentedness.