
\chapter{Introduction}
\label{ch:1}

“Foreign accent” is usually considered an issue of perception, rather than production. Only  perceivable deviations in non-native (L2) speech are considered features of “foreign accent.” As \citet[p.160]{Munro_1998} defines it, foreign accent is “\textit{the extent to which an L2 learner’s speech is perceived to differ from native speaker norms.}” Foreign accent has been a widely discussed issue in the literature, and consonant manner of articulation (e.g., \citealp{Riney_1999, Solon_2015}), vowel quality (e.g., \citealp{Hahn_2004, Munro_2001, Zielinski_2008}) and various prosodic elements (e.g., \citealp{Hahn_2004, Munro_2001, Zielinski_2008}) have been found to greatly affect foreign accent perception. However, there seems to be no consensus on which elements contribute more to foreign accentedness. The current study aims to explore the degree of foreign accentedness exhibited by various phonetic/phonological patterns in L2 speech. Specifically, we investigate which phonetic/phonological aspects of L2 English speech are perceived as more saliently foreign-sounding by native (L1) listeners of American English. The current study also endeavors to uncover possible reasons for why some phonetic and phonological patterns in L2 speech are perceived as more foreign-sounding than others. The focus of the current study is L1 and L2 English speech. The term “accent” used in this dissertation refers specifically to L1 and/or L2 English speech patterns. The cited studies, unless otherwise noted, are research on L1 and/or L2 English speech.

The current study investigates both phonetic and phonological differences between L1 and L2 English speech. For example, using [æ̞] instead of [æ] could be an issue of phonetics, while structural changes such as vowel epenthesis (e.g.,  [æskə] for “\textit{ask}”) could be an issue of phonology. However, we consider all these differences an issue of phonetics, because we assume that the L2 speakers who produced these stimuli were aware of and were trying to mimic L1 productions. In other words, we assume that the L2 speakers know how the phonetic segments should be organized in English (i.e. phonology). We consider the differences between the L2 speech samples and their corresponding L1 productions as a reflection of the differences between L1 and L2 speakers' articulatory configurations, which should be discussed in terms of phonetics. Therefore, the term ``phonetic pattern" is used throughout the dissertation to refer to pronunciation differences between L1 and L2 speech. We fully acknowledge that some of the phonetic patterns discussed in this dissertation could be considered issues of phonology. 

\section{Background}

According to \citet{Wells_1982}, a difference between varieties of English may involve any or all aspects of the language (e.g., syntax, pronunciation, and lexicon etc.), while a difference between accents of English is restricted to pronunciation. Since the current study aims to investigate the accentedness of various speech patterns, the focus is on the phonetic and phonological aspects of L1 and L2 utterances. 

It should be noted that an accent is not unique to L2 speakers. It is widely accepted among linguists that an accent is something both L1 and L2 speakers have \citep{Lippi-Green_2012}. However, L1 and L2 accents could lead to different sociolinguistic consequences. People with an L1 accent are often perceived by other L1 speakers as being more trustworthy and capable than people with an L2 accent \citep{Gluszek_2010}. L2 accents, although bearing no relationship to one’s intelligence or personal character, are sometimes viewed as characteristics of ineptitude \citep{Gluszek_2010}. L2 speakers, therefore, place great importance on the accuracy of their pronunciations \citep{Waniek-Klimczak_2015}.

Other than the differences in sociolinguistic consequences, L1 and L2 English accents also ex- hibit different types of phonetic/phonological patterns \citep{Lippi-Green_2012}. Take t/d-deletion in English as an example. L1 English speakers are more likely to delete /t/ or /d/ when they are past tense morphemes (e.g., /d/ in “called,” /t/ in “packed”) than when they are part of the stem of a word (e.g., /d/ in “hold,” /t/ in “pact”) \citep{Guy_1991}. L2 speakers’ t/d-deletion strategy, on the other hand, does not seem to concern whether the   /t/ or /d/ is part of the stem of a word  \citep{Edwards_2011, Hansen_2004}.

 The differences between L1 and L2 English accents could also be discussed from a perceptual perspective. Previous research found that children as young as seven years old are capable of distinguishing L1 accents from L2 ones \citep{Floccia_2009}. They are, however, not as successful at distinguishing other regional L1 accents from their own \citep{Floccia_2009}. The reason for such asymmetry was attributed to the type of phonetic variations in L1 and L2 speech. As \citet{Floccia_2009} found, L2 speech exhibits greater distortions of consonants than L1 speech. Children in \citet {Floccia_2009} seemed to have relied more on consonant information when distinguishing between L1 and L2 accents. This suggests that foreign-accented speech exhibits characteristics that distinguish itself from L1 speech varieties and these characteristics are perceivable. 

In the United States, General American English (GA) is often colloquially referred to as the “neutral American accent” or  the “standard American accent” \citep{Lippi-Green_2012, Wells_1982}. However, GA is not a uniform accent \citep{Wells_1982}. According to \citet{Wells_1982}, the rime of the word “square” in GA could have at least three L1 variations, namely [ɛɹ], [æɹ] and [eɪɹ]; for words such as “star,” the vowel could range from a fronted [a] to a back [ɑ]. In other words, the GA is not one specific accent of American English, but a continuum of accents. Although dictionaries and English textbooks tend to teach the “Standard American English,” no such standardized English accent has existed in reality \citep{Lippi-Green_2012}. 

The heterogeneity among L1 English accents poses a great challenge to empirical inquiries on the differences between L1 and L2 accents. When discussing the hypothetical “native speaker norms” or a “typical native accent,” previous research tends to rely on mean acoustic measurements of L1 speech produced by a specific group of L1 speakers (e.g., \citealp{Chan_2016, McCullough_2013}). For example, mean L1 voice onset time (VOT) was used to represent the native speaker norm of aspiration length for plosive consonants \citep{Riney_1999}. Mean formant frequencies were used to represent L1 vowel qualities \citep{Chan_2016}. 

With the assumption that mean acoustic measurements represent the native speaker norms, previous perception research has identified several phonetic correlates of foreign accent (e.g., \citealp{Chan_2016,Riney_1999, Solon_2015}). Consonant manner of articulation, VOT, and vowel epenthesis have often been found to be closely associated with perceived foreign accent \citep{Chan_2016, Magen_1998, Munro_1998, Riney_1998, Solon_2015}. Several studies also suggested that vowel quality contributes to foreign accent perception \citep{Braun_2011, Major_1986,Park_2013}. Various prosodic elements such as intonation and speech rate have also been found to greatly affect foreign accent perception \citep{Hahn_2004, Kang_2010, Munro_2001, Zielinski_2008}.

There is no doubt that the degree of perceived foreign accent is affected by various phonetic pat- terns. However, it is difficult to determine a hierarchy of their importance and it is equally challeng- ing to quantify their relative impact on accentedness detection \citep{Munro_1995, Rognoni_2014}. Among the few studies that did evaluate the relative impact of different phonetic patterns, the findings were often inconsistent and sometimes contradicted one another \citep{Magen_1998, van_den_Doel_2006}. Some of the problems in these studies could be due to experimental design rather than strictly linguistic factors. For example, the stimuli of several previous studies were acoustically edited or synthesized (e.g., \citealp{Chan_2016, Jilka_2000, Magen_1998}. It is possible that acoustic manipulations in one dimension (e.g., intonation) could have affected the perception of acoustic signals in other dimensions (e.g., vowel height) \citep{Whalen_1995}. Therefore, an acoustically edited or synthesized speech sample might not be representative of natural speech. 

Some research placed L2 stimuli in carrier sentences without controlling for the effect of phonological context \citep{Magen_1998, van_den_Doel_2006}. It is possible that a given L2 stimulus could be perceptually accented in one context but not accented in another, which could potentially explain conflicting results found by previous studies. Further, regarding the training of listeners specifically for the task of judging accents, some studies claim that a training session is necessary to familiarize listeners with the task (e.g., \citealp{Major_1986}), while others argue that a training session could introduce biases that affect listeners' judgments (e.g., \citealp{McDermott_1986}). However, there seems to be a paucity of literature that directly compares accentedness judgments between trained and untrained listeners. The current study addresses this methodological gap by using unaltered stimuli and conducting two separate experiments which directly compared trained and untrained listeners. 

While some phonetic patterns in L2 speech were found to be perceptually more foreign accented than others, the reason for such a phenomenon is not readily clear (e.g., \citealp{Magen_1998}). Previous research has often made ad hoc claims that listeners’ knowledge of English (e.g., English phonotactics) is one of the underlying mechanisms that affects accentedness judgments on L2 speech (e.g., \citealp{Park_2013}). The current study investigates this claim directly by approximating L1 English listeners’ knowledge of English phonetics and phonology (L1 knowledge) with a computational model. L2 speech samples were evaluated by the model to generate dissimilarity scores that approximated the degree of difference between L1 and L2 speech samples. Accentedness ratings on the L2 speech samples were compared against the dissimilarity scores to observe how exactly L1 knowledge affects accentedness perception.

\section{The Current Study}

The current study focuses on which phonetic patterns in L2 speech are perceptually more accented to L1 listeners of American English, rather than how acoustic cues correlate with perceived foreign accentedness. For example, the current study investigates whether an L2 pronounciation of “\textit{thick}” as [θik] is perceptually more foreign-accented to an L1 listener than pronouncing “\textit{thick}” as [θɪk]. The degree of acoustic differences between an L2 [i] and its L1 target [ɪ] and their effect on  accentedness perception are not a focus of the current study. 

In lieu of acoustic differences between L1 and L2 productions, the current study opts to focus directly on perception data. To this end, phonetically trained personnel were recruited to phonetically transcribe L1 and L2 utterances of English using the International Phonetic Alphabet (IPA). IPA symbols represent transcribers’ perceptions. Once a sound is transcribed with an aspiration symbol [ʰ] (e.g., [pʰ]), it is safe to conclude that the transcriber perceived it.  Furthermore, when several transcribers perceive the same phonetic phenomenon, the perception is corroborated. 

The current study conducted two perception experiments to elicit accentedness judgments from L1 listeners of American English. One hundred L2 audio speech samples were selected from the Speech Accent Archive (SAA: \citealp{Weinberger_saa_2019}) as stimuli for the two experiments. The stimuli were selected based on their IPA transcriptions in the SAA. In classifying the 100 L2 audio stimuli, the current study surveyed productions of 100 L1 speakers of American English. The most common productions among the 100 L1 speakers were considered L1 target productions. For example, 90\% of the L1 productions for the word “\textit{thick}” were rendered as [θɪk]. [θɪk] was therefore selected as the L1 target production for “\textit{thick}.” Such a treatment was based on the assumption that L1 English listeners should all be familiar with the most common L1 productions and consequently consider them as containing no foreign accent. L2 speech samples that were transcribed the same as their L1 target production (e.g. [θɪk] for “\textit{thick}”) were termed “match” stimuli, meaning that the L2 productions matched their L1 target productions. L2 productions that did not match their L1 target productions were termed “mismatch” stimuli. Both match and mismatch stimuli were produced by L2 English speakers. 

The current study focuses specifically on segmental and syllable structural aspects of L2 English speech. Therefore, the mismatches were defined in terms of segmental and syllable structural characteristics. Prosodic elements such as intonation and speech rhythm were ignored in the process of stimuli selection. Prosody of the stimuli was controlled for by selecting stimuli without lexical stress shifts. Intonational and durational information was controlled for computationaly with an alignment algorithm, which estimates intonational and durational differences between L1 and L2 speech samples. Section \ref{ch4:prosody} in Chapter \ref{ch:4} discusses this method in more detail.

It should be noted that some of the mismatch productions, although they were not transcribed the same as their L1 target productions, are not unique to L2 speakers of English. For example, two native speakers of American English in the SAA pronounced “\textit{thick}” as [t̪ɪk], which does not match the most common L1 production [θɪk]. [t̪ɪk] was termed by the current study as a mismatch stimulus, simply because it does not match the most common L1 production. Presumably, mismatch stimuli with phonetic patterns that exist in L1 speech would be judged as less accented. On the other hand, some mismatch stimuli contain phonetic patterns that do not occur in L1 speech (e.g., pronouncing “\textit{ask}” as [æskə]). These stimuli presumably carry a relatively higher degree of accentedness. 

The mismatch stimuli were further divided into three groups based on three types of mismatches, namely, stimuli with consonant mismatches (e.g.,  [tɪk] for “\textit{thick}”), stimuli with vowel mismatches (e.g., [θik] for “\textit{thick}”), and stimuli with syllable structure mismatches (e.g., [æskə] for “\textit{ask}”). The current study therefore investigated four types of stimuli (i.e., the match stimuli and the three types of mismatch stimuli). L1 listeners of American English (i.e., the raters) were recruited to rate the foreign ccentedness of the four types of stimuli. Results show that the mismatch stimuli were judged as being more accented than the match stimuli. Among the three types of mismatch stimuli, stimuli with consonant and syllable mismatches were judged as being the most accented, and stimuli with vowel mismatches were judged as the least accented. In addition, the frequency of occurrences of a phonetic pattern in L1 speech potentially affects accentedness judgment. Chapter \ref{ch:4}, Chapter \ref{ch:5} and Chapter \ref{ch:6} of this dissertation discuss the findings in more detail. 


To investigate how raters’ L1 phonetic and phonological knowledge (L1 knowledge) affects their accentedness judgment, the current study computationally constructed an L1 production model based on IPA transcriptions from 100 L1 speakers of American English. The L1 production model was a matrix of “association strengths,” approximating the probable occurrence of a certain word based on a certain sequence of phonetic segments. For example, the model approximated the probability of the segment sequence [pʰli] occurring as a prediction of the word “\textit{please}.”  The model could be intuitively understood as the probability that a rater believed that the hearing of the [pʰli] portion of the word resulted in the intended meaning of “\textit{please}.” The L1 production was therefore a matrix of “association strengths,” mapping phonetic segment sequences to lexical meanings.

IPA transcriptions of the 100 L2 stimuli were compared against the L1 production model to generate dissimilarity scores, representing the degree of difference between an L2 stimulus and the 100 L1 productions. Dissimilarity scores and acccentedness ratings of the L2 stimuli were compared against each other to evaluate whether the dissimilarity scores could predict the degree of foreign accentedness. The results show that the dissimilarity scores indeed correlate significantly with accentedness ratings, indicating that L1 knowledge, as approximated by the L1 production model, could have significantly affected accentedness judgments. The L1 production model and the subsequent comparisons between L1 and L2 speech samples provide insights into the nature of L1 knowledge and how it affects accentedness perception. Chapter \ref{ch:6} of this dissertation provides further details.


\section{Rationale}

The current study hypothesizes that some phonetic patterns in L2 speech are perceptually more accented than others. Specifically, stimuli with consonant mismatches are perceptually more accented than stimuli with vowel mismatches. This hypothesis is grounded in previous studies on lexical identification and speech perception  (e.g., \citealp{Kronrod_2012, Nespor_2003}). These studies often claim that consonants are more important in lexical identification and are generally more susceptible to categorical perception. These two potential attributes of consonants could have affected the accentedness of  stimuli with consonant mismatches.

Speech processing involves the ability to detect statistical regularities in the input \citep{Romberg_2010}. Both adults and children have been shown to be able to parse speech by extracting statistical regularities such as transitional probabilities between phonetic segments \citep{Romberg_2010}. However, not all statistical regularities are equally weighted in every circumstance. \citet{Nespor_2003} suggest that there is a division of labor between consonants and vowels in language processing and acquisition. Transitional probabilities between consonants are more important at the lexical level, while transitional probabilities of vowels are more important at prosodic and syntactic levels \citep{Nespor_2003}. The distortion of consonants would therefore be more likely to impair lexical identification, while the distortion of vowels would affect the processing of prosody and syntax. If lexical identification is a component of perceived foreign accentedness, then consonant mismatches, especially the ones that do not represent L1 dialectal variations, would be perceived as more accented than vowel mismatches. Alternatively, if prosodic and syntactic information weighs more heavily than lexical identification in accentedness judgment, then vowel mismatches would perhaps be more accented than consonant mismatches. 

Consonants and vowels not only differ in their potential functions in lexical identification, but also in how they are perceived. Research on speech perception showed that phonemic differences between consonants are more easily perceived as a categorical difference than phonemic differences between vowels \citep{Altmann_2014, Kronrod_2012}. Perception of consonants, especially obstruent consonants, is relatively more categorical, while the perception of vowels is relatively more continuous \citep{Altmann_2014, Kronrod_2012}. Therefore, phonemic alternations of consonants are probably more accented, because such alternations are more easily perceived as a categorical difference. 

The difference between consonants and vowels could also be discussed from phonological-distributional perspectives. Unlike consonants, vowels are cross-linguistically fewer in number and more prone to lose their distinctiveness \citep{Nespor_2003}. If vowel variations are indeed more common than consonant variations in L1 speech, then vowel variations in L2 speech could be more tolerable than consonant variations. Consequently, stimuli with vowel mismatches could be perceived as less accented than stimuli with consonant mismatches. 

Studies and theories on lexical identification and speech sound categorization do not provide much evidence for the accentedness of stimuli with syllable mismatches, such as segment deletion and segment epenthesis. However, since segment epenthesis is less common than segment deletion in L1 English speech \citep{Johnson_2004a}, it is possible that stimuli with segment deletion is less accented than stimuli with segment epenthesis.


\section{Organization of the Dissertation}

This dissertation proceeds as follows: Chapter \ref{ch:2} begins with a literature review on certain phonetic/phonological aspects of foreign accents, theories of language perception, and experimental methodologies. Chapter \ref{ch:2} summarizes major findings in previous studies and discuss their advantages and shortcomings.  Chapter \ref{ch:3} discusses how the stimuli for the experiments were selected. 

Chapter \ref{ch:4} discusses Experiment 1, which was conducted as a pilot study. Major findings and potential methodological shortcomings are discussed. Accentedness rankings of phonetic patterns in L2 speech are provided. Discussions in this chapter center on the potential reasons for the observed accentedness ratings. Chapter \ref{ch:5} considers Experiment 2, which addresses the potential methodological problems in Experiment 1. In addition to major findings of the experiment, discussions in this chapter considers how the methodological differences between Experiment 1 and Experiment 2 affected accentedness judgment. 

Chapter \ref{ch:6} discusses Experiment 3. The aim of Experiment 3 is to investigate whether raters’ L1 knowledge affects their accentedness judgment. A computational model was adopted to construct an L1 production model, which approximated the degree of phonetic and phonological differences between L1 and L2 speakers' speech samples. The dissimilarity scores generated by the model were compared against accentedness ratings from Experiment 2 to investigate how L1 knowledge affects accentedness judgment. This chapter further discusses the successes and failures of the model in approximating accentedness judgment. 

Chapter \ref{ch:7} considers implications of the findings, problems of the current study, and offers suggestions for future research.


